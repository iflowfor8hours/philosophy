\documentclass[letterpaper,12pt]{article}
\usepackage{amssymb}
\usepackage{amsmath}

\setlength{\voffset}{-1 in}
\addtolength{\textheight}{2.15 in}

\renewcommand{\implies}{\rightarrow}
\newcommand{\pr}{\text{pr}}
\newcommand{\Prob}{\text{Pr}}
%\newcommand{\PR}{\text{PR}}
\newcommand{\dom}{\mathrm{dom\ }}
\newcommand{\rng}{\mathrm{rng\ }}


\newcommand{\A}{\mathfrak{A}}
\newcommand{\N}{\mathbb{N}}

\renewcommand{\O}{\mathfrak{O}}
\newcommand{\T}{\mathfrak{T}}
\newcommand{\proves}{\vdash}

\renewcommand{\phi}{\varphi}

\newtheorem{definition}{Definition}
\newtheorem{claim}{Claim}

\begin{document}
\section{Propositional logic}
These are the operators of \emph{classical propositional logic}:
\begin{enumerate}
\item
$\land$ means ``and'' (e.g., $a \land b$ is ``a and b are both true'')
\item
$\lor$ means ``or'' ($a \lor b$ is ``a or b is true'', i.e., ``at least one of a or b is true'')
\item
$\neg$ means ``not'' ($\neg a$ is ``a is not true'')
\item
$p \implies q$ means ``if p, then q'', equivalently, ``either p is false or q is true'', equivalently $\neg p \lor q$
\end{enumerate}

Exercises:
\begin{enumerate}
\item
Assume the sky is blue and the moon is made of rocks. Is the sentence ``if the sky is green, then the moon is made of blue cheese'' true or false?
\item
If $p \implies q$ is true, is $q \implies p$ (the \emph{converse}) always true?
\item
If $p \implies q$ is true, is $\neg q \implies \neg p$ (the \emph{contrapositive}) always true?
\item
Express $a \lor b$ using only $\land$ and $\neg$. (This is called \emph{De Morgan's law}. It shows that $\land$ and $\neg$ are sufficient to express all of propositional logic.)
\item
The operator $\uparrow$ has these semantics: $a \uparrow b$ means ``either a is false or b is false''. Express all the other operators using $\uparrow$. (This operator is called \emph{NAND} or the \emph{Sheffer stroke}.)
\end{enumerate}

\section{First-order logic}
Read $\forall$ as ``for every'' and $\exists$ as ``there exists'.

Example: let $H(x, y)$ denote ``x cuts y's hair''. Then we can write ``there is someone who cuts everyone's hair`` as:
$$\exists x \forall y H(x, y)$$
and ``everyone has someone who cuts their hair'' as:
$$\forall x \exists y H(y, x)$$

Notice that $\exists x \phi(x)$ is equivalent to $\neg \forall x \neg \phi(x)$. (This relies on the assumption that at least one thing exists.)

Exercises:
\begin{enumerate}
\item
Which one of these sentences implies the other? Are they equivalent?
\item
Write the negations of both sentences.
\item
Write the definition of limit using logical symbols. Then write its negation.
\end{enumerate}
\end{document}